\documentclass[11pt,a4paper,english]{article}
\usepackage[english]{babel} % Using babel for norwegian hyphenation
\usepackage{lmodern} % Changing the font
\usepackage[utf8]{inputenc}
\usepackage[T1]{fontenc}

\usepackage{tcolorbox}
\usepackage[parfill]{parskip} % Removes indents
\usepackage{amsmath} % Environment, symbols etc...
\usepackage{amssymb}
\usepackage{float} % Fixing figure locations
\usepackage{multirow} % For nice tables
\usepackage{graphicx} % For pictures etc...
\usepackage{enumitem} % Points/lists
\usepackage{url}
\usepackage{hyperref}

\definecolor{red}{RGB}{255,10,10}

% To include code(-snippets) with æøå
\usepackage{listings}
\lstset{
language=c++,
showspaces=false,
showstringspaces=false,
frame=l,
}

\tolerance = 5000 % Bedre tekst
\hbadness = \tolerance
\pretolerance = 2000

\numberwithin{equation}{section}

\newcommand{\conj}[1]{#1^*}
\newcommand{\ve}[1]{\mathbf{#1}} % Vektorer i bold
\let\oldhat\hat
\renewcommand{\hat}[1]{\mathbf{\oldhat{#1}}}
\newcommand{\trans}[1]{#1^\top}
\newcommand{\herm}[1]{#1^\dagger}
\newcommand{\Real}{\mathbb{R}}
\newcommand{\bigO}[1]{\mathcal{O}\left( #1 \right)}

\newcommand{\spac}{\hspace{5mm}}

\title{UNIK4660 - Assignment 1}
\author{Wilhelm Karlsen and Magnus Eden}
\date{\today}

\begin{document}
\maketitle

\section{Problem 1}
\begin{align}
	u &= -y \nonumber \\
	v &=  x \nonumber \\
	w &= zt \nonumber
\end{align}
We are given the velocity field above, in the right handed Cartesian coordinate system. Using this field, we are to show that, at $t = 0$, the stream lines are circles centered on the origin. 
\\
This can be done by using the dot products algebraic definition,
$$ \bar{V} \bullet \bar{x} = \bar{V}_{1}\bar{x}_{1} + \bar{V}_{2}\bar{x}_{2} + \bar{V}_{3}\bar{x}_{3}$$
where $\bar{V}$ is the velocity vector and $\bar{x}$ is the position vector from the origin.
\\
\\
As $u$ and $v$ are independent of time and $w = 0$ when $t = 0$, we can ignore the third dimension in this problem.

\begin{align*}
	\bar{V} \bullet \bar{x} &= \bar{V}_{1}\bar{x}_{1} + \bar{V}_{2}\bar{x}_{2}	\\
	\bar{V} \bullet \bar{x} &= -yx + xy	\\
	\bar{V} \bullet \bar{x} &= 0
\end{align*}
This means that the velocity vector is always orthogonal to the position, which would be the angular velocity.

?Further proff/explanation required?


\section{Problem 2}
We are to show that the streamlines are circles with center at the z-axis going through the point $(r, 0, z_{0})$.
We are given the definition of $r$ as $r^2 = x^2 + y^2$, and given $tan \theta = \frac{y}{x}$.
The streamlines can then be expressed by:
\begin{align}
	\frac{dx}{d\theta} &= -y \nonumber \\
	\frac{dy}{d\theta} &=  x \nonumber \\
	\frac{dz}{d\theta} &= zt \nonumber
\end{align}

As this problem is considered at $t=0$, the $z$ dimension is always 0 and can be ignored.
This then becomes a two dimensional problem and we can use the definition of $r$ to find the position.
\begin{align*}
x &= \sqrt{r^2 - y^2}	& 
y &= \sqrt{r^2 - x^2}
\end{align*}

This can then be inserted into the streamline equations, and since we are only concerned with the point $(r, 0, z_{0})$, we get:
\begin{align*}
	\frac{dx}{d\theta} &= -\sqrt{r^2 - x^2} &= -\sqrt{r^2 - r^2} &= 0 	\\
	\frac{dy}{d\theta} &=  \sqrt{r^2 - y^2} &= -\sqrt{r^2 - 0^2} &= r	\\
	\frac{dz}{d\theta} &=  0
\end{align*}

?What was the reason this was proof enough!

No streamline that starts at $z > 0$ can cross $z = 0$ due to the ?

\section{Problem 3}
We are to find the formula for the streamline, $SL$, going through the point $P = (r, 0, w_{0})$, when the time $t > 0$. \\
Using the result from problem 3, we see that the x and y 'elements' are independent of time and so the only change is that the $z$-dimension is no longer always 0. 
We can write the formula as:
$$ SL(x, y, z) = -\sqrt{r^2 - x^2} + \sqrt{r^2 - y^2} + zt$$
and for the point $P$ this becomes:
$$ SL(r, 0, w_{0}) = -\sqrt{r^2 - r^2} + \sqrt{r^2 - 0^2} + w_{0}t$$
where r is the radius of the circle around on which z lies.
\\
The streamlines going through $(r, 0, 0)$ are the same as the ones found in problem 3, since when $z=0$ the $z$ 'element' becomes independent of time.

\section{Problem 4}
We are to find the path line, $PL$ for the velocity field, $\bar{u}$, given in problem 1, that goes through the point $P = (r, 0, z_{0})$ at $t=0$.
\\
The path lines are defined to be the trajectories of an individual particle in a fluid.
Where the suffix $P$ is indicates a single particle and not the point $P$.
\[ 
\begin{cases} 
   \frac{d\bar{x}}{dt} &= \bar{u}_{P}(\bar{x}_{P}, t)	\\
   \bar{x}_{P}(t_{0})  &= \bar{x}_{P0}  
\end{cases} 
\]
As $t=0 = t_{0}$, the path line is simply the position of the particle at $t=0$. To prove this we need to derive it from the velocity field at the specified point.
\begin{align*}
	\frac{dx}{dt} &= \bar{u_1} = -y 	\\	
	dx &= -ydt						\\
	\int_0^x dx &= \int_0^t -ydt	\\
	x &= x_0 - yt
\end{align*}

\begin{align*}
	\frac{dy}{dt} &= \bar{u_2} = x 	\\	
	dy &= xdt						\\
	\int_0^y dy &= \int_0^t xdt	\\
	y &= xt + y_0
\end{align*}

\begin{align*}
	\frac{dz}{dt} 			&= \bar{u_3} = zt 			\\	
	\frac{dz}{z} 			&= tdt						\\
	\int_0^y \frac{1}{z}dz 	&= \int_0^t tdt				\\
	ln(\frac{z}{z_0}) 		&= \frac{1}{2}t^2			\\
	z &= z_0 e^{\frac{1}{2}t^2}
\end{align*}

The path line can then be written as:
$$PL(x, y, z, t) = (x_0 - yt)\hat{i} + (xt+y_0)\hat{j} + (z_0 e^{\frac{1}{2}t^2})\hat{k}$$
The path line on point $P$ at As $t=0$ becomes:
$$PL(r, 0, z_0, 0) = (x_0)\hat{i} + (y_0)\hat{j} + (z_0)\hat{k}$$

\section{Problem 5}
We are to find the expression for the streak lines emerging from the point $(r_0, 0, z_0)$. \\

Streak lines are the curve connecting all particles that pass through a given point, $p$, from time $t_0 \leq s \leq t$, and is defined as $x = x(\xi[p,s],t)$.

We derive an expression for the spatial coordinates from the velocity field given in the same way as we did in problem 4.
\begin{align*}
	\frac{dx}{dt} &= u	\\
	x(x,y,z, t) &= (x_0-yt)\hat{i} + (y_0+xt)\hat{j} + (z_0e^{\frac{1}{2}t^2})\hat{k}
\end{align*}
We can replace the initial coordinates, $x_0$, $y_0$ and $z_0$, with the material coordinates $\xi_i$, since $\xi$ is equal to the initial spatial coordinates for each particle in the flow.
$$x(x,y,z, t) = (\xi_1-yt)\hat{i} + (\xi_2+xt)\hat{j} + (\xi_3e^{\frac{1}{2}t^2})\hat{k}$$
Reshuffling this gives us:
\begin{equation} \label{eq:material}
\xi_i = (x+yt)\hat{i} + (y-xt)\hat{j} + (ze^{-\frac{1}{2}t^2})\hat{k}
\end{equation}

The expression for the particle at position $p(x,y,z)$ and time $s$ is then:
\begin{equation} \label{eq:positionP}
\xi_i = (x+ys)\hat{i} + (y-xs)\hat{j} + (z^{-\frac{1}{2}s^2})\hat{k}
\end{equation}
The streak line expression is then found by inserting ~\eqref{eq:material} into ~\eqref{eq:positionP} and using the coordinates for the emerging point, $(r_o, 0, z_0)$:
\begin{align*}
	(x+yt)\hat{i} + (y-xt)\hat{j} + (ze^{-\frac{1}{2}t^2})\hat{k}
		= 
	(x+ys)\hat{i} + (y-xs)\hat{j} + (z^{-\frac{1}{2}s^2})\hat{k}
\end{align*}
\begin{align*}
	x_1 &= r_0 	& 	
	x_2 &= r_0t-r_0s	&	
	x_3 &= z_0e^{\frac{1}{2}t^2 - \frac{1}{2}s^2}
\end{align*}
$$ Streak line = r_0\hat{i} + (r_0t-r_0s)\hat{j} + (z_0e^{\frac{1}{2}t^2 - \frac{1}{2}s^2})\hat{k} $$

\section{Problem 6}
Using the velocity field given in problem 1, we are to calculate the vorticity field.
The vorticity field is defined as $\bigtriangledown \times \bar{v}$.
This gives us:
\begin{align*}
	\bigtriangledown \times \bar{v} &= 		
		\begin{vmatrix}
			\bar{i} 	 & \bar{j} 		& \bar{k}			 		\\
  			\frac{d}{dx} & \frac{d}{dy}	& \frac{d}{dz}				\\
  			-y 			 & x 			& zt			 		
		\end{vmatrix}
		\\
		&=  (\frac{dzt}{dy}-\frac{dx}{dz})\bar{i} + 
			(\frac{dy}{dz}-\frac{dzt}{dx})\bar{j} + 
			(\frac{dx}{dx}+\frac{dy}{dy})\bar{k}
		\\
		Vorticity &= 0\bar{i} + 0\bar{j} + 2\bar{k}
\end{align*}

\section{Problem 7}

Since 

\[ \frac{\partial u_{i}}{\partial t_{j}}
   = \frac{1}{2} \left( \frac{\partial u_i}{\partial x_j}
      + \frac{\partial u_j}{\partial y_i}
       \right) +
       \frac{1}{2} \left( \frac{\partial u_i}{\partial x_j}
      - \frac{\partial u_j}{\partial y_i}
       \right) = 
       s_{ij} + \frac{1}{2}\omega_{ij}\]
       
and 

\[\omega_{ij}= 
\frac{1}{2} \left( \frac{\partial u_i}{\partial x_j}
      - \frac{\partial u_j}{\partial y_i}
       \right)\]
       
we can find the strain rate tensor by simply expanding the tensor formula and solving for it using our two vectors given by the velocity field
\[\vec{x}= <x, y, z>\] \[\vec{u}=<-y, x, zt>\]

\[s_{ij}=
\begin{bmatrix}
       \frac{\partial -y}{\partial x} & 
       \frac{1}{2}(\frac{\partial x}{\partial x} + \frac{\partial -y}{\partial y}) & 
       \frac{1}{2}(\frac{\partial zt}{\partial x} + \frac{\partial -y}{\partial z}) \\[0.3em]
       
       \frac{1}{2}(\frac{\partial -y}{\partial y} + \frac{\partial x}{\partial x}) & 
       \frac{\partial x}{\partial y} & 
       \frac{1}{2}(\frac{\partial zt}{\partial y} + \frac{\partial x}{\partial z}) \\[0.3em]
       
       \frac{1}{2}(\frac{\partial -y}{\partial z} + \frac{\partial zt}{\partial x}) & 
       \frac{1}{2}(\frac{\partial x}{\partial z} + \frac{\partial zt}{\partial y}) & 
       \frac{\partial zt}{\partial z} \\[0.3em]
     \end{bmatrix}
     =\begin{bmatrix}
       0 & 
       0 & 
       0\\[0.3em]
       
       0 & 
       0 & 
       0\\[0.3em]
       
       0 & 
       0 & 
       t\\[0.3em]
     \end{bmatrix}\]
     
The rotation tensor can similarly be found using the same method.     
     
     \[\omega_{ij}=
\begin{bmatrix}
       \frac{\partial -y}{\partial x} - \frac{\partial -y}{\partial x} & 
       \frac{1}{2}(\frac{\partial x}{\partial x} - \frac{\partial -y}{\partial y}) & 
       \frac{1}{2}(\frac{\partial zt}{\partial x} - \frac{\partial -y}{\partial z}) \\[0.3em]
       
       \frac{1}{2}(\frac{\partial -y}{\partial y} - \frac{\partial x}{\partial x}) & 
       \frac{\partial x}{\partial y} - \frac{\partial x}{\partial y} & 
       \frac{1}{2}(\frac{\partial zt}{\partial y} - \frac{\partial x}{\partial z}) \\[0.3em]
      
       \frac{1}{2}(\frac{\partial -y}{\partial z} - \frac{\partial zt}{\partial x}) & 
       \frac{1}{2}(\frac{\partial x}{\partial z} - \frac{\partial zt}{\partial y}) & 
       \frac{\partial zt}{\partial z} - \frac{\partial zt}{\partial z} \\[0.3em]
     \end{bmatrix}
     =\begin{bmatrix}
       0 & 
       2 & 
       0\\[0.3em]
       
       -2 & 
       0 & 
       0\\[0.3em]
       
       0 & 
       0 & 
       0\\[0.3em]
     \end{bmatrix}\]

\section{Problem 8}
We are to calculate the relative dilatation following a particle path.

\end{document}